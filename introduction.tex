\chapter*{\textsc{Introduction générale}}
\lettrine{A}{pparu} dans le grand \emph{Larousse de la langue française} qu'en 1972, le terme \textbf{Environnement}\footnote{Ensemble des élements naturels ou artificiels qui conditionnent la vie de l'homme} est recent. de surcroît l'homme et ses activités impactent, malencontreusement,  pas toujours en bien sur la santé de son environnement. d'où le protocole de \emph{Kyoto}  en 1997 qui visais la limitation des emissions de gaz à effet de serre et l'accord de parie en 2015 qui vise au maintien de la l'augmentation de la température mondiale bien en dessous de 2 degrés Centigrade et de mener des efforts encore plus poussés pour limiter l'augmentation de la température à 1.5 degré centigrade au-dessus des niveaux pré industriels.\\

Une des inquiétudes majeures de ce début de siècle est la pollution atmosphérique et son impact sur l’environnement et la santé humaine. si depuis plus de 25 ans, la pollution due à l’industrie a baissé de 45 
à $65$, celle due aux transports a quand a elle a augmenté de plus de 30 , et pour principale cause l'accroissement du trafic automobile


La pollution de l'environnement est aujourd'hui un sujet délicat en  sur la santé de la planète. La pollution est étroitement liée à la consommation de l'énergie,  

