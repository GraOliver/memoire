\documentclass[oneside]{book}
\usepackage[UTF8]{inputenc}
\usepackage{graphicx}
\usepackage[T1]{fontenc}
\usepackage[francais]{babel}
\usepackage[a4paper,margin=1in]{geometry}

\usepackage{color}
\definecolor{titre}{RGB}{0,121,181}
\newcommand{\couleurT}{\color{titre}}

%%%% Fancy chapeter
\usepackage[Glenn]{fncychap}
\usepackage{Lettrine}
\usepackage{amsmath,amsfonts,amsthm,geometry,lipsum}
\usepackage{hyperref}

%%%% couleur
\usepackage{xcolor,tcolorbox,tikz}
\usepackage{graphicx}    % Pour gérer les images
%% Bibiographie
%\usepackage[style=authoryear]{biblatex}
%%%%% Proprieté des liens dans le texte
\hypersetup{
    colorlinks =true,
    linkcolor=black,
    urlcolor=black,
    runbordercolor=black,
    runcolor=black,
    citecolor=black,
}
%% debut du doncument

\begin{document}
%%%/////////////////////// Page de Garde ///////////////////////////////////////
\include{PG}
%% Page des garde

\color{black}
\tableofcontents
\listoffigures
\listoftables
%%//////////////////////////////////////// CHAPITRE 1 /////
\include {"Chapitre1"}
%%!!!!!!!!!!!!!!!!!!!!! chapitre 2
\include{"Chapitre2"}
%%///////// Création de la bibliographie
\nocite{*}
\bibliographystyle{apalike}
\bibliography{bibliographie}

\vspace{2cm}
%%% Annexe

\end{document}
